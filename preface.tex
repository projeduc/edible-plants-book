%    This file is part of Edible Food Book, a CC-BY-SA-3.0 book.
%    Copyright (C) 2013  Abdelkrime Aries <kariminfo0@gmail.com>
%
%    This program is free software: you can redistribute it and/or modify
%    it under the terms of the GNU General Public License as published by
%    the Free Software Foundation, either version 3 of the License, or
%    any later version.
%
%    This program is distributed in the hope that it will be useful,
%    but WITHOUT ANY WARRANTY; without even the implied warranty of
%    MERCHANTABILITY or FITNESS FOR A PARTICULAR PURPOSE.  See the
%    GNU General Public License for more details.
%
%    You should have received a copy of the GNU General Public License
%    along with this program.  If not, see <http://www.gnu.org/licenses/>.

\newpage
\hrule
\phantomsection
\addcontentsline{toc}{part}{Preface}
\begin{flushright}
{\huge \textbf{Preface}}
\end{flushright}
\vspace{1cm}

This is a book about Edible plants, which are the plants we can use in our daily dishes. 
It contains some plants categorized as their culinary use: fruits, nuts, cereals, legumes, etc. 
The main idea, here, is to provide the reader with the names of these plants using many languages, illustrated by pictures.
So, this book is destined for language learners and travelers.

I got the idea of making such a book from contacting a friend of mine from Japan (Miho san). 
Sometimes, we were talking about some edible plants like "Prickly pear" which I didn't know its name in English that time. 
I tried to translate those plants names from Arabic to English then to Japanese, and it was so hard.
So, I decided to make a book that can help me and other Arabic-Japanese learners. 
I thought may be the book will be more informative if I add some other languages.
In a result, I added two other languages I can understand: English and French.

I hope this book will be helpful for you. 
My only request from you is to spread this book, so can people make use of it.
Thank you so much for reading my book.

\newpage
\hrule
\begin{flushleft}
{\huge \textbf{\AR{\textbattar{مقدمة}}}}
\end{flushleft}
\vspace{1cm}

\begin{flushright}
\AR{\textdimnah{ %
هذا كتاب عن النباتات الصالحة للأكل، وهي النباتات التي يمكننا استخدامها في الأطباق اليومية.
الكتاب يحتوي على بعض النباتات المصنفة على أساس الاستخدام في الطهي: الفواكه والمكسرات والحبوب والبقوليات، إلخ.
الفكرة الرئيسية، هنا، هي تزويد القارئ بأسماء هذه النباتات مستخدمين في ذلك العديد من اللغات، وتوضيحات باستخدام الصور.
لذلك، هذا الكتاب موجه لمتعلمي اللغات والمسافرين.
}}

\AR{\textdimnah{ %
جاءتني فكرة إصدار هذا الكتاب من محادثاتي مع صديقة لي من اليابان (ميهو سان).
في بعض الأحيان، كنا نتحدث عن بعض النباتات الصالحة للأكل مثل \textquotedblright التين الشوكي\textquotedblleft\ الذي لم أكن أعرف اسمه باللغة الإنجليزية حينها.
حاولت أن أترجم أسماء تلك النباتات من العربية إلى الإنجليزية ثم إلى اليابانية، وكان هذا صعبا جدا.
لذا، قررت أن أؤلف كتابا يمكن أن يساعدني وغيري من متعلمي العربية واليابانية.
اعتقدت أن الكتاب سيكون أكثر إفادة في حالة إضافة بعض اللغات الأخرى.
نتيجة لذلك، أضفت اثنتين من اللغات التي أستطيع فهمها: الإنجليزية والفرنسية.
}}

\AR{\textdimnah{ %
آمل أن يكون هذا الكتاب ذا فائدة بالنسبة لكم.
طلبي الوحيد منكم هو نشر هذا الكتاب، حتى يتمكن الناس من الاستفادة منه.
شكرا جزيلا لكم على قراءة كتابي.
}}
\end{flushright}